\documentclass[12pt]{ctexart}
\usepackage{amsmath,amssymb,bm}
\usepackage[a4paper,margin=1in]{geometry}
\usepackage{hyperref}
\numberwithin{equation}{section}

\title{拉格朗日与有限元相关对话整理}
\author{}
\date{}

\begin{document}
\maketitle
\tableofcontents

\section*{说明}
本文件整理自 \texttt{latex/lag.tex} 中的对话内容,按问答形式归纳,统一为 LaTeX 格式,去除交互提示语,仅保留核心问题与回答。\par
涉及主题包括:变分定义与方向导数、张量冒号内积、连续到离散的过渡、有限元弱形式、时间离散(显式/隐式)、非线性与切线刚度矩阵、动力学方程推导等。\\
符号约定:$\Omega$ 为当前域,$\Omega_0$ 为参考域;$\boldsymbol{\varphi}$ 为位移映射,$\mathbf{u}$ 为位移;$\mathbf{F}=\nabla_X \boldsymbol{\varphi}$ 为形变梯度;$\mathbf{P}=\partial\psi/\partial\mathbf{F}$ 为第一 Piola--Kirchhoff 应力;$E=\tfrac12(\mathbf{F}^\top\mathbf{F}-\mathbf{I})$ 为 Green--Lagrange 应变;冒号“:”为 Frobenius 内积 $A:B=\mathrm{tr}(A^\top B)$。

\section{问答}

\subsection{变分与一阶导数的写法}
\textbf{问:} 先写对 $G$ 的偏导:
\begin{align}
\frac{\partial\psi}{\partial G_{00}} &= 2\mu\, G_{00} + \lambda\,t, & \frac{\partial\psi}{\partial G_{11}} &= 2\mu\, G_{11} + \lambda\,t, & \frac{\partial\psi}{\partial G_{01}} &= 4\mu\, G_{01},
\end{align}
其中 $t=\operatorname{tr} G=G_{00}+G_{11}$。再用 $G$ 关于 $f$ 的导数:
\begin{align}
\delta G_{00} &= f_0\cdot \delta f_0, & \delta G_{11} &= f_1\cdot \delta f_1, & \delta G_{01} &= \tfrac12\,(f_1\cdot \delta f_0 + f_0\cdot \delta f_1).
\end{align}
得到对 $f_0,f_1$ 的一阶导(列向量形式):
\begin{align}
\frac{\partial\psi}{\partial f_0} &= f_0(2\mu\, G_{00} + \lambda\, t) + f_1(2\mu\, G_{01}),\\
\frac{\partial\psi}{\partial f_1} &= f_0(2\mu\, G_{01}) + f_1(2\mu\, G_{11} + \lambda\, t).
\end{align}
这里求导为什么要写变分?\\
\textbf{答:} 当 $\psi=\psi(F)$ 是关于张量(矩阵)$F$ 的标量泛函时,\emph{变分}给出方向导数的定义:
\[
\delta\psi = \frac{\partial \psi}{\partial F} : \delta F = \mathbf{P} : \delta \mathbf{F},
\]
其中 $\mathbf{P}=\partial\psi/\partial \mathbf{F}$ 是第一 Piola--Kirchhoff 应力;冒号“:”为 Frobenius 内积,
\[
A:B = \operatorname{tr}(A^\top B) = \sum_{i,j} A_{ij} B_{ij}.
\]
该关系在每个空间点上成立(点态线性化)。若需总能量变分,需对体积积分:
\[
\delta W = \int_{\Omega} \mathbf{P}(x) : \delta \mathbf{F}(x)\, dV.
\]

\subsection{变分的定义与冒号的含义}
\textbf{问:} 定义导数时变分的定义是什么?式中的冒号“:”是什么意思?这些量不是连续场吗?\\
\textbf{答:} 变分是沿某一方向的无穷小扰动引起的无穷小变化(Gâteaux 导数):
\[
\delta \psi[F;\delta F] := \lim_{\epsilon\to 0} \frac{\psi(F+\epsilon\,\delta F)-\psi(F)}{\epsilon}.
\]
在矩阵情形下,$\delta\psi = P:\delta F$,其中“:”是点态定义的张量内积。由于 $P,F$ 是\emph{连续场},总变分需对体积分以累加各点贡献:
\[
\delta W = \int_{\Omega} \mathbf{P}(x):\delta \mathbf{F}(x)\, dV.
\]

\subsection{连续场到离散内积的过渡}
\textbf{问:} 两个张量场是连续的,如何进行离散的求和/内积?\\
\textbf{答:} 点态关系 $\delta\psi(x)=\mathbf{P}(x):\delta \mathbf{F}(x)$ 通过体积分得到总变分。有限元离散将积分近似为单元加权求和:
\[
\int_{\Omega} \mathbf{P} : \delta \mathbf{F}\, dV \approx \sum_e \int_{\Omega_e} \mathbf{P} : \delta \mathbf{F}\, dV \approx \sum_e V_e\, (\mathbf{P}_e : \delta \mathbf{F}_e),
\]
线性三角单元内 $\nabla N_i$ 为常数,故 $\mathbf{F}$ 在单元内为常量,从而单元能量 $E_e = V_e\,\psi(\mathbf{F}_e)$。

\subsection{张量场的维度与有限维近似}
\textbf{问:} 张量场的维度是无穷吗?\\
\textbf{答:} 是的,$\mathbf{F}:\Omega\to\mathbb{R}^{3\times 3}$ 是函数空间中的元素(如 $[H^1(\Omega)]^{3\times 3}$),属于无限维。有限元用形函数近似到有限维:
\[
\boldsymbol{\varphi}(X) \approx \sum_a N_a(X)\, \mathbf{u}_a,\quad \mathbf{F}(X)=\nabla_X\boldsymbol{\varphi}(X) \approx \sum_a (\nabla_X N_a)\, \mathbf{u}_a.
\]
积分型内积在连续空间为
\[
\langle P,Q\rangle = \int_{\Omega} P:Q\, dV,
\]
离散后转为单元求和。

\subsection{单元内常量近似与高斯点}
\textbf{问:} 连续情况下,单元内是否假定 $F$ 为常数?单元要剖分到多小?可以只包含一个点吗?\\
\textbf{答:} 连续理论下 $\mathbf{F}(X)$ 逐点变化;数值上,线性单元内 $\nabla_X N_i$ 为常数,故 $\mathbf{F}$ 分片常量。单元尺寸 $h$ 越小近似越好。数值积分常用高斯点:线性单元用一个积分点即可代表单元常量场的能量:
\[
E_e \approx w\, \psi(\mathbf{F}(X_g)).
\]
单元缩为单点体积为零(能量为零),因此“一个点”仅作为积分近似代表单元体积加权,不是逐点积分。

\subsection{为什么不能逐点积分}
\textbf{问:} 能量积分能否逐点进行?\\
\textbf{答:} 连续理论中,单点体积 $dV\to 0$,单点贡献为零,积分需在非零体积上进行。离散中用积分点加权近似单元积分:
\[
E \approx \sum_e \sum_{g} w_g\, \psi(F(X_g)).
\]

\subsection{形变梯度:逐点定义与单元一致近似}
\textbf{问:} 理论上形变梯度逐点定义,但数值上必须在每个小体积内一致吗?\\
\textbf{答:} 理论上 $F(X)=\nabla\boldsymbol{\varphi}(X)$ 逐点定义;线性单元中因 $\nabla N_i$ 常数,$F$ 在单元内一致(常量)。高阶单元则在积分点上采样近似其变化。

\subsection{Eulerian 与 Lagrangian 弱形式对比}
\textbf{问:} 欧拉描述下需要乘测试函数得到足够方程;拉格朗日描述下如何操作?\\
\textbf{答:} 二者操作一致:乘测试函数、体积分、分部积分、离散化。区别在积分域与主变量。拉格朗日下参考域 $\Omega_0$ 固定,主变量为位移 $\mathbf{u}(X,t)$:
\[
\int_{\Omega_0} w_i\, \rho_0 \ddot{\mathbf{u}}\, dV = \int_{\Omega_0} w_i (\nabla_X\cdot \mathbf{P})\, dV + \int_{\Omega_0} w_i \mathbf{b}_0\, dV,
\]
分部积分得
\[
\int_{\Omega_0} \nabla_X w_i : \mathbf{P}\, dV - \int_{\partial\Omega_0} w_i (\mathbf{P}\cdot \mathbf{N})\, dA.
\]

\subsection{拉格朗日描述下的时间离散}
\textbf{问:} 在拉格朗日描述下继续时间离散如何进行?\\
\textbf{答:} 空间离散得节点方程:
\[
\mathbf{M}\, \ddot{\mathbf{u}} + \mathbf{f}_{\mathrm{int}}(\mathbf{u}) = \mathbf{f}_{\mathrm{ext}}.
\]
显式中心差分:
\[
\ddot{\mathbf{u}}^n \approx \frac{\mathbf{u}^{n+1}-2\mathbf{u}^n+\mathbf{u}^{n-1}}{\Delta t^2},\quad \mathbf{u}^{n+1} = 2\mathbf{u}^n - \mathbf{u}^{n-1} + \Delta t^2\, \mathbf{M}^{-1}(\mathbf{f}_{\mathrm{ext}}^n - \mathbf{f}_{\mathrm{int}}^n).
\]
隐式 Newmark($\gamma=\tfrac12,\, \beta=\tfrac14$)需牛顿迭代解非线性方程:
\[
\mathbf{M}\, \ddot{\mathbf{u}}^{n+1} + \mathbf{f}_{\mathrm{int}}(\mathbf{u}^{n+1}) = \mathbf{f}_{\mathrm{ext}}^{n+1}.
\]

\subsection{离散方程各项的含义}
\textbf{问:} 解释 $\mathbf{M}\, \ddot{\mathbf{u}}^{n+1} + \mathbf{f}_{\mathrm{int}}(\mathbf{u}^{n+1}) = \mathbf{f}_{\mathrm{ext}}^{n+1}$ 中各项含义。\\
\textbf{答:}
\begin{itemize}
  \item $\mathbf{M}$:质量矩阵,来源于 $\int_{\Omega_0} N_i\, \rho_0\, N_j\, dV$,与节点惯性相关。显式算法常用对角化(lumped mass)。
  \item $\ddot{\mathbf{u}}^{n+1}$:节点加速度,由时间离散公式给出(如中心差分或 Newmark)。Newmark 下有 $\partial \ddot{\mathbf{u}}^{n+1}/\partial \mathbf{u}^{n+1} = 1/(\beta\Delta t^2)$。
  \item $\mathbf{f}_{\mathrm{int}}(\mathbf{u}^{n+1})$:内部力,$\displaystyle \int_{\Omega_0} \nabla_X N_i : \mathbf{P}(\mathbf{u})\, dV$,由材料应力与几何非线性决定,通常随位移非线性。
  \item $\mathbf{f}_{\mathrm{ext}}^{n+1}$:外力,$\displaystyle \int_{\Omega_0} N_i \, \mathbf{b}_0\, dV + \int_{\partial\Omega_0} N_i (\mathbf{P}\cdot \mathbf{N})\, dA$(边界项可由自然边界条件给出与 $\mathbf{P}\cdot\mathbf{N}$ 一致的名义牵引)。
  \item 残差:$\mathbf{R}(\mathbf{u}^{n+1}) = \mathbf{M}\, \ddot{\mathbf{u}}^{n+1} + \mathbf{f}_{\mathrm{int}}(\mathbf{u}^{n+1}) - \mathbf{f}_{\mathrm{ext}}^{n+1}$,隐式迭代令其趋近 0。
\end{itemize}

\subsection{非线性与切线刚度矩阵}
\textbf{问:} 为什么内部力是非线性的?什么是切线刚度矩阵?\\
\textbf{答:} 几何非线性(如 $E=\tfrac12(\mathbf{F}^\top \mathbf{F} - \mathbf{I})$ 含二次项)与材料非线性都会使 $\mathbf{f}_{\mathrm{int}}(\mathbf{u})$ 成为非线性函数。以 St.
Venant--Kirchhoff 为例:$\psi=\mu(E:E)+\tfrac{\lambda}{2}(\operatorname{tr}E)^2$,则 $\mathbf{P}=\mathbf{F}\,\mathbf{S}$,$\mathbf{S}=\partial\psi/\partial E=2\mu E+\lambda(\operatorname{tr}E)\mathbf{I}$。牛顿迭代线性化残差:
\[
\mathbf{u}^{k+1} = \mathbf{u}^k - \mathbf{J}(\mathbf{u}^k)^{-1}\, \mathbf{R}(\mathbf{u}^k),\quad \mathbf{J}=\frac{\partial \mathbf{R}}{\partial \mathbf{u}} = \frac{\mathbf{M}}{\beta\,\Delta t^2} + \mathbf{K}_{\mathrm{tan}},
\]
其中 $\mathbf{K}_{\mathrm{tan}}=\partial \mathbf{f}_{\mathrm{int}}/\partial \mathbf{u}$ 为切线刚度矩阵,表示微小位移增量引起的内部力变化率。线性弹性小变形下 $\mathbf{K}_{\mathrm{tan}}$ 退化为常规线性刚度矩阵。

\subsection{动力学方程的推导}
\textbf{问:} 动力学方程从何而来?\\
\textbf{答:} 由牛顿第二定律与应力作用推导。在参考配置 $\Omega_0$,对任意体积:
\[
\int_{\Omega_0} \rho_0 \ddot{\mathbf{u}}\, dV = \int_{\partial\Omega_0} \mathbf{P}\cdot \mathbf{N}\, dA + \int_{\Omega_0} \mathbf{b}_0\, dV.
\]
散度定理给出内力体积形式 $\int_{\Omega_0} (\nabla_X\cdot P)\, dV$,故逐点形式为
\[
\rho_0 \ddot{\mathbf{u}} = \nabla_X\cdot \mathbf{P} + \mathbf{b}_0.
\]
弱形式乘测试函数并分部积分,离散后得到
\[
\mathbf{M}\, \ddot{\mathbf{u}} + \mathbf{f}_{\mathrm{int}}(\mathbf{u}) = \mathbf{f}_{\mathrm{ext}}.
\]

\end{document}