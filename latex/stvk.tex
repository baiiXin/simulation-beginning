% !TeX program = xelatex
\documentclass[11pt]{article}
\usepackage[UTF8]{ctex} % 中文内容建议使用 XeLaTeX/LuaLaTeX 编译
\usepackage{amsmath,amssymb,bm}
\usepackage{mathtools}
\usepackage{geometry}
\geometry{a4paper, margin=1in}
\usepackage{hyperref}
\usepackage{listings}

% 便捷记号
\newcommand{\Ithree}{\mathbf{I}_3}
\newcommand{\tr}{\operatorname{tr}}

\title{St. Venant--Kirchhoff (StVK) 三角单元 Hessian 的规范推导}
\author{\small 符号化推导与实现对照}
\date{}

\begin{document}
\maketitle

\section*{目的与概览}
将三角单元(3 个顶点,每顶点 3 自由度)的 StVK 能量到位置空间的二阶导数(Hessian)做出完整、符号化、逐步的推导,并给出便于实现的块矩阵表达式与伪代码。文档按数学推导(变分)顺序排列,便于学术写作与审稿附件。

\section{记号与约定}
\begin{itemize}
  \item 顶点索引:$a,b\in\{0,1,2\}$。顶点位置 $x_a\in\mathbb{R}^3$。单元自由度按顶点排列:$[x_0,x_1,x_2]$。
  \item 变形梯度 $F\in\mathbb{R}^{3\times 2}$ 写作列向量:$F=[f_0\, f_1]$,其中 $f_0,f_1\in\mathbb{R}^3$。
  \item 参考到当前的线性映射(逆形变梯度 $\mathrm{Dm}^{-1}$)分量:$D_{00},D_{01},D_{10},D_{11}$。定义局部边向量:$x_{01}=x_1-x_0,\ x_{02}=x_2-x_0$。
  \item 单元面积(或面积权重)记为 $A$(如需积分乘以面积)。
  \item StVK 材料参数:$\mu,\lambda$。单位矩阵:$\Ithree$。外积:$u\otimes v = u v^\top$。
\end{itemize}

\section{从位置到 \texorpdfstring{$F$}{F} 的线性关系}
\begin{align}
f_0 &= x_{01} D_{00} + x_{02} D_{10} = - (D_{00}+D_{10}) x_0 + D_{00} x_1 + D_{10} x_2,\\
f_1 &= x_{01} D_{01} + x_{02} D_{11} = - (D_{01}+D_{11}) x_0 + D_{01} x_1 + D_{11} x_2.
\end{align}
于是对任意顶点 $a$,有
\begin{equation}\label{eq:dfdx_coeff}
\frac{\partial f_i}{\partial x_a} = \alpha^i_a\, \Ithree,\qquad i\in\{0,1\},\ a\in\{0,1,2\},
\end{equation}
其中系数
\begin{align}
\alpha^0 &= [\alpha^0_0,\alpha^0_1,\alpha^0_2] = [-(D_{00}+D_{10}),\ D_{00},\ D_{10}],\\
\alpha^1 &= [\alpha^1_0,\alpha^1_1,\alpha^1_2] = [-(D_{01}+D_{11}),\ D_{01},\ D_{11}].
\end{align}
这是线性的关键——$\partial f_i/\partial x_a$ 是标量乘单位矩阵,因此拉回到位置空间时非常简洁。

\section{Green 应变与能量密度}
定义 Green 应变分量:
\begin{align}
G_{00} &= \tfrac12 (f_0\cdot f_0 - 1),&
G_{11} &= \tfrac12 (f_1\cdot f_1 - 1),&
G_{01} &= \tfrac12 (f_0\cdot f_1).
\end{align}
写成矩阵形式:$G=\tfrac12(F^\top F - I_2)$,$G$ 为对称 $2\times2$ 矩阵。
\\
StVK 单位体能量密度:
\begin{equation}
\psi = \mu\,\|G\|_F^2 + \tfrac12\lambda\,(\tr G)^2,
\end{equation}
其中 $\|G\|_F^2 = G_{00}^2 + G_{11}^2 + 2 G_{01}^2$,$\tr G = G_{00}+G_{11}$。

\section{一阶导:$\partial\psi/\partial f_i$(PK1 的列)}
先写对 $G$ 的偏导:
\begin{align}
\frac{\partial\psi}{\partial G_{00}} &= 2\mu\, G_{00} + \lambda\,t, &
\frac{\partial\psi}{\partial G_{11}} &= 2\mu\, G_{11} + \lambda\,t, &
\frac{\partial\psi}{\partial G_{01}} &= 4\mu\, G_{01},
\end{align}
其中 $t=\tr G=G_{00}+G_{11}$。再用 $G$ 关于 $f$ 的导数:
\begin{align}
\delta G_{00} &= f_0\cdot \delta f_0, &
\delta G_{11} &= f_1\cdot \delta f_1, &
\delta G_{01} &= \tfrac12\,(f_1\cdot \delta f_0 + f_0\cdot \delta f_1).
\end{align}
得到对 $f_0,f_1$ 的一阶导(列向量形式):
\begin{align}
\frac{\partial\psi}{\partial f_0} &= f_0(2\mu\, G_{00} + \lambda\, t) + f_1(2\mu\, G_{01}),\\
\frac{\partial\psi}{\partial f_1} &= f_0(2\mu\, G_{01}) + f_1(2\mu\, G_{11} + \lambda\, t).
\end{align}

\section{二阶导:$\partial^2\psi/\partial f_i\partial f_j$($f$ 空间的 Hessian)}
我们寻找 $3\times3$ 矩阵块 $\mathcal{A}_{ij}$,使得对任意微分 $\delta f_0,\delta f_1$:
\begin{equation}
\delta^2\psi = \begin{bmatrix}\delta f_0^\top & \delta f_1^\top\end{bmatrix}
\begin{bmatrix}\mathcal{A}_{00} & \mathcal{A}_{01} \\
\mathcal{A}_{10} & \mathcal{A}_{11}\end{bmatrix}
\begin{bmatrix}\delta f_0 \\ \delta f_1\end{bmatrix}.
\end{equation}
将 $\psi = \mu\,\|G\|_F^2 + \tfrac12\lambda\, t^2$ 分别考察两部分的二阶导。

\subsection{来自 $\lambda$ 的贡献}
直接计算可得(对称形式):
\begin{align}
\mathcal{A}_{00}^\lambda &= \lambda\,(f_0\otimes f_0), &
\mathcal{A}_{11}^\lambda &= \lambda\,(f_1\otimes f_1), &
\mathcal{A}_{01}^\lambda &= \lambda\,(f_0\otimes f_1), &
\mathcal{A}_{10}^\lambda &= (\mathcal{A}_{01}^\lambda)^\top = \lambda\,(f_1\otimes f_0).
\end{align}
解释:$t=G_{00}+G_{11}$ 关于 $f$ 的梯度与自身成比例,因此 $\partial^2 (\tfrac12\lambda t^2)$ 给出外积结构。

解释更详尽地如下。设
\[
\psi_\lambda = \tfrac12 \lambda \, t^2,\qquad t = G_{00}+G_{11}= \tfrac12\bigl(\|f_0\|^2+\|f_1\|^2\bigr)-1.
\]
由链式法则,
\[
\frac{\partial\psi_\lambda}{\partial f_0}= \lambda t\,\frac{\partial t}{\partial f_0}= \lambda t\,f_0,\qquad
\frac{\partial\psi_\lambda}{\partial f_1}= \lambda t\,\frac{\partial t}{\partial f_1}= \lambda t\,f_1.
\]
再次对 $f$ 求导并用 $\tfrac{\partial t}{\partial f_0}=f_0,\ \tfrac{\partial t}{\partial f_1}=f_1$,可得
\begin{align}
\frac{\partial^2\psi_\lambda}{\partial f_0\partial f_0} &= \lambda\bigl(f_0\otimes f_0 + t\,\Ithree\bigr),\\
\frac{\partial^2\psi_\lambda}{\partial f_1\partial f_1} &= \lambda\bigl(f_1\otimes f_1 + t\,\Ithree\bigr),\\
\frac{\partial^2\psi_\lambda}{\partial f_0\partial f_1} &= \lambda\,(f_0\otimes f_1),\\
\frac{\partial^2\psi_\lambda}{\partial f_1\partial f_0} &= \lambda\,(f_1\otimes f_0).
\end{align}
其中各对角块中的各向同性项 $\lambda t\,\Ithree$ 与随后 $\mu$ 部分得到的对应项一并归纳进公式 \eqref{eq:c_def} 的 $c\,\Ithree$,故本小节所给的 $\lambda$ 项仅保留外积结构,与上式一致。


\subsection{来自 $\mu$ 的贡献}

详细推导如下。记
\[
\psi_\mu = \mu\,\|G\|_F^2 \;=\; \mu\bigl(G_{00}^2 + G_{11}^2 + 2G_{01}^2\bigr).
\]
设 $t_{00}=G_{00},\;t_{11}=G_{11},\;t_{01}=G_{01}$ 以简化记号,则
\begin{equation}
\mathrm{d}\psi_\mu = 2\mu\,(t_{00}\,\mathrm{d}t_{00}+t_{11}\,\mathrm{d}t_{11}+2t_{01}\,\mathrm{d}t_{01}).
\end{equation}
利用
\[
\mathrm{d}t_{00}=f_0\!\cdot\!\mathrm{d}f_0,\quad
\mathrm{d}t_{11}=f_1\!\cdot\!\mathrm{d}f_1,\quad
\mathrm{d}t_{01}=\tfrac12\bigl(f_1\!\cdot\!\mathrm{d}f_0+f_0\!\cdot\!\mathrm{d}f_1\bigr),
\]
可得一阶导(与前一节结果一致):
\begin{align}
\frac{\partial\psi_\mu}{\partial f_0} &= 2\mu\bigl(t_{00}f_0 + t_{01}f_1\bigr), &
\frac{\partial\psi_\mu}{\partial f_1} &= 2\mu\bigl(t_{01}f_0 + t_{11}f_1\bigr).
\end{align}
再对 $f$ 求导:
\begin{align}
\frac{\partial^2\psi_\mu}{\partial f_0\partial f_0} &= 2\mu\Bigl[(f_0\cdot f_0)\,\Ithree + 2\,(f_0\otimes f_0) + (f_1\otimes f_1)\Bigr],\\
\frac{\partial^2\psi_\mu}{\partial f_1\partial f_1} &= 2\mu\Bigl[(f_1\cdot f_1)\,\Ithree + 2\,(f_1\otimes f_1) + (f_0\otimes f_0)\Bigr],\\
\frac{\partial^2\psi_\mu}{\partial f_0\partial f_1} &= 2\mu\Bigl[(f_0\cdot f_1)\,\Ithree + (f_1\otimes f_0)\Bigr],\\
\frac{\partial^2\psi_\mu}{\partial f_1\partial f_0} &= 2\mu\Bigl[(f_0\cdot f_1)\,\Ithree + (f_0\otimes f_1)\Bigr].
\end{align}
为了与文献及实现中的惯用记法一致,本文将整体因子 $2\mu$ 折半写作 $\mu$,于是得到下列表达式:

对 $\|G\|_F^2$ 展开并整理(略去冗长代数,给出结果与物理解释):
\begin{align}
\mathcal{A}_{00}^\mu &= \mu\big( (f_0\cdot f_0)\, \Ithree + 2(f_0\otimes f_0) + (f_1\otimes f_1) \big),\\
\mathcal{A}_{11}^\mu &= \mu\big( (f_1\cdot f_1)\, \Ithree + 2(f_1\otimes f_1) + (f_0\otimes f_0) \big),\\
\mathcal{A}_{01}^\mu &= \mu\big( (f_0\cdot f_1)\, \Ithree + (f_1\otimes f_0) \big),\\
\mathcal{A}_{10}^\mu &= (\mathcal{A}_{01}^\mu)^\top = \mu\big( (f_0\cdot f_1)\, \Ithree + (f_0\otimes f_1) \big).
\end{align}
此外还有一个标量乘 $\Ithree$ 的合并项(对应实现中的 \texttt{two\_dpsi\_dIc}):
\begin{equation}\label{eq:c_def}
c := -\mu + \big(\tfrac12 I_c - 1\big)\,\lambda,\qquad I_c = f_0\cdot f_0 + f_1\cdot f_1.
\end{equation}
这个 $c$ 项乘以 $\Ithree$ 需要被加入到 $\mathcal{A}_{00}$ 与 $\mathcal{A}_{11}$:
\begin{align}
\mathcal{A}_{00} &\leftarrow \mathcal{A}_{00} + c\, \Ithree, &
\mathcal{A}_{11} &\leftarrow \mathcal{A}_{11} + c\, \Ithree.
\end{align}

\subsection{合并得到 $\mathcal{A}_{ij}$ 的闭式}
\begin{align}
\mathcal{A}_{00} &= \lambda (f_0\otimes f_0) + c\, \Ithree + \mu\big( (f_0\cdot f_0)\, \Ithree + 2(f_0\otimes f_0) + (f_1\otimes f_1)\big), \label{eq:A00}\\
\mathcal{A}_{01} &= \lambda (f_0\otimes f_1) + \mu\big( (f_0\cdot f_1)\, \Ithree + (f_1\otimes f_0)\big), \label{eq:A01}\\
\mathcal{A}_{11} &= \lambda (f_1\otimes f_1) + c\, \Ithree + \mu\big( (f_1\cdot f_1)\, \Ithree + 2(f_1\otimes f_1) + (f_0\otimes f_0)\big). \label{eq:A11}
\end{align}
上述表达式与实现中 \texttt{d2E\_dF2\_00/01/11} 相对应。

\section{将 $F$ 空间的 Hessian 拉回到位置空间(块矩阵 $H_{ab}$)}
链式法则(块形式):
\begin{equation}\label{eq:block_chain}
H_{ab} = \sum_{i,j\in\{0,1\}} \left(\frac{\partial f_i}{\partial x_a}\right)^\top \mathcal{A}_{ij} \left(\frac{\partial f_j}{\partial x_b}\right).
\end{equation}
利用 \eqref{eq:dfdx_coeff} 中的 $\partial f_i/\partial x_a = \alpha^i_a \Ithree$,得到简洁形式:
\begin{equation}\label{eq:Hab}
H_{ab} = \alpha^0_a\alpha^0_b\,\mathcal{A}_{00} + \alpha^1_a\alpha^1_b\,\mathcal{A}_{11} + \alpha^0_a\alpha^1_b\,\mathcal{A}_{01} + \alpha^1_a\alpha^0_b\,\mathcal{A}_{01}^\top.
\end{equation}
这是对每个顶点对 $(a,b)$ 的 $3\times3$ 块矩阵。将所有 $H_{ab}$ 按行列放入即可得到完整的 $9\times9$ Hessian。理论上 $H_{ab}=H_{ba}^\top$,数值实现应显式保证对称性(例如最后做 $H=(H+H^\top)/2$)。

\section{部分块展开举例}
令 $a=0$,写出 $H_{00},H_{01},H_{02}$:
\begin{align}
H_{00} &= (\alpha^0_0)^2\,\mathcal{A}_{00} + (\alpha^1_0)^2\,\mathcal{A}_{11} + \alpha^0_0\alpha^1_0\,\mathcal{A}_{01} + \alpha^1_0\alpha^0_0\,\mathcal{A}_{01}^\top,\\
H_{01} &= \alpha^0_0\alpha^0_1\,\mathcal{A}_{00} + \alpha^1_0\alpha^1_1\,\mathcal{A}_{11} + \alpha^0_0\alpha^1_1\,\mathcal{A}_{01} + \alpha^1_0\alpha^0_1\,\mathcal{A}_{01}^\top,\\
H_{02} &= \alpha^0_0\alpha^0_2\,\mathcal{A}_{00} + \alpha^1_0\alpha^1_2\,\mathcal{A}_{11} + \alpha^0_0\alpha^1_2\,\mathcal{A}_{01} + \alpha^1_0\alpha^0_2\,\mathcal{A}_{01}^\top.
\end{align}

\section{实现伪代码(可直接映射到 \texttt{evaluate\_stvk\_force\_hessian})}
\begin{lstlisting}[language={},basicstyle=\ttfamily\small]
# 1. 计算 f0, f1
f0 = x01 * D00 + x02 * D10
f1 = x01 * D01 + x02 * D11

# 2. 基本内积与应变
f0f0 = dot(f0,f0); f1f1 = dot(f1,f1); f0f1 = dot(f0,f1)
G00 = 0.5*(f0f0 - 1); G11 = 0.5*(f1f1 - 1); G01 = 0.5*f0f1
t = G00 + G11
Ic = f0f0 + f1f1
c = -mu + (0.5*Ic - 1.0)*lambda

# 3. 构造 A_ij (3x3 矩阵) 按 (A00),(A01),(A11)
A00 = lambda*(f0 outer f0) + c*I3 + mu*( f0f0*I3 + 2*(f0 outer f0) + (f1 outer f1) )
A01 = lambda*(f0 outer f1) + mu*( f0f1*I3 + (f1 outer f0) )
A11 = lambda*(f1 outer f1) + c*I3 + mu*( f1f1*I3 + 2*(f1 outer f1) + (f0 outer f0) )

# 4. 计算 alpha 系数
alpha0 = [-(D00+D10), D00, D10]
alpha1 = [-(D01+D11), D01, D11]

# 5. 组装 H_ab 块
for a in 0..2:
  for b in 0..2:
    H_ab = alpha0[a]*alpha0[b]*A00 + alpha1[a]*alpha1[b]*A11 
          + alpha0[a]*alpha1[b]*A01 + alpha1[a]*alpha0[b]*A01^T
    place H_ab into global 9x9 at block (a,b)  # each H_ab is 3x3

# 6. 可选:乘以面积 A。
H *= A
# 7. 强制对称化: H = 0.5*(H + H^T)
\end{lstlisting}

\section{数值与数学注意事项}
\begin{itemize}
  \item 当应变非常小时($\|G\|\to 0$),可用线性化近似以降低计算量,或如代码所示阈值返回零以避免奇异。
  \item 确保数值对称性(末尾对称化步骤)。
  \item 若要加入阻尼或质量矩阵影响(如隐式时间积分),需在此基础上再加相应项;本文件只覆盖弹性 StVK 部分。
\end{itemize}

\section*{变量对应(对照 Python/Warp 实现)}
\begin{itemize}
  \item \texttt{f0} $\leftrightarrow$ $f_0$; \texttt{f1} $\leftrightarrow$ $f_1$。
  \item \texttt{DmInv00/01/10/11} $\leftrightarrow$ $D_{00},D_{01},D_{10},D_{11}$。
  \item \texttt{df0\_dx}, \texttt{df1\_dx} 源自 $\alpha^i_a$ 与顶点选择(mask)的组合。
\end{itemize}

\end{document}